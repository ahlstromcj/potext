%-------------------------------------------------------------------------------
% usage
%-------------------------------------------------------------------------------
%
% \file        usage.tex
% \library     Documents
% \author      Chris Ahlstrom
% \date        2024-03-19
% \update      2024-04-14
% \version     $Revision$
% \license     $XPC_GPL_LICENSE$
%
%     Provides a description of how to use potext in an application.
%
%-------------------------------------------------------------------------------

\section{Potext Usage in Applications}
\label{subsec:potext_usage}

   This section briefly covers the usage of \textsl{Potext} in an
   application.
   A real sample is included in \texttt{library/tests/hellopotext.cpp}
   (see \sectionref{subsec:potext_hello_potext}).
   A small sample application showing the usage of
   \textsl{Potext} as a \textsl{Meson subproject} in a completely
   separate application project is
   contained in:

   \begin{verbatim}
      extras/code/mini-potext-test.tar.xz
   \end{verbatim}

   Unpack this file in its own directory and check it out.

\subsection{Main Module Using Potext}
\label{subsubsec:potext_usage_main}

   The first thing is to add the following header file
   to the module defining the \texttt{main()} function.

   \begin{verbatim}
      #include "po/potext.hpp"      // includes "po/gettext.hpp"
   \end{verbatim}

   For clarity, \texttt{potext.hpp} is better, but it does include
   an extra header file.

   If \textsl{Potext} support is optional for the project, then do
   something like this; \texttt{PROJECT\_USE\_POTEXT}
   is a macro optionally defined when configuring the project build.

   \begin{verbatim}
      #if defined PROJECT_USE_POTEXT
      #include "po/potext.hpp"   // includes "po/gettext.hpp"
      else
      #define _(str)             (str)
      #define N_(str)            str
      #endif
   \end{verbatim}

   An application using "gettext" internationaliztion generally needs
   to call \texttt{setlocale()},
   \texttt{textdomain()}, and \texttt{bindtextdomain()}.
   The following function declared in the \texttt{gettext.hpp}
   header file wraps up these functions in one call.

   \begin{verbatim}
      std::string init_app_locale
      (
          const std::string & arg0,
          const std::string & pkgname,
          const std::string & domainname,
          const std::string & dirname,
          const std::wstring & wdirname = L"",
          int category = (-1)
      )
   \end{verbatim}

   \texttt{arg0}.
   The path-name by which the program was called. This information
   can determine more precisely where installed \texttt{.po} files might be
   stored.

   \texttt{pkgname}.
   The name of the PACKAGE, which can be the short name for
   the program, such as "helloworld".

   \texttt{domainname}.
   The base name of a message catalog, such as "en\_US".
   It must consist of characters legal in filenames.
%  We might support a value of "0" to reset the domain to
%  "messages".
   An application might want to use its
   package name, such as "helloworld", for the domain name.
   If empty, then the environment variable \texttt{TEXTDOMAIN} is used.
   If that's empty, then \texttt{LC\_ALL} is used.
   If that's empty, then \texttt{LC\_MESSAGE} is used.
   Lastly, if that's empty, then \texttt{LANG} is used.

   \texttt{dirname}.
   Provides the name of the \texttt{LOCALEDIR}.
   The standard search directory is \texttt{/user/share/locale}.
   If empty, then the environment variable
   \texttt{TEXTDOMAINDIR} is used.
   If the name is "user", then the \texttt{.po} files are searched
   for in \texttt{/home/user/.config/package/} or
   \texttt{C:/Users/user/AppData/Local}, instead of some
   place in the system.
  
   \texttt{wdirname}.
   The wide-string version of the locale directory name, most useful
   in \textsl{Windows}.
   The use of the wide-string parameter is optional,
   and not well tested yet.

   \texttt{category}.
   The area that is covered, such as
   \texttt{LC\_ALL}, \texttt{LC\_MONETARY}, and \texttt{LC\_NUMERIC}.
   The default value selects \texttt{LC\_ALL}.

   The following calls are made for the setup:

   \begin{enumber}
      \item \texttt{std::setlocale()} sets the application's current
         category and locale.
         The category is \texttt{LC\_ALL} by default, and the locale
         is empty, so that the locale parts are modified according to
         environment variables.
      \item \texttt{po::set\_locale\_info()} sets up the domain name
         and the locale directory name. If empty, the
         environment variables discussed above are used.
         In addition, it is determined if the locale directory
         is a system directory, the user's configuration directory,
         or some arbitrary directory containing \texttt{.po} files.
      \item \texttt{po::init\_lib\_locale()} first
         asks the dictionary-manager (see 
         \sectionref{sec:potext_architecture})
         to add all of the dictionaries (po files) in that directory
         to the list of selectable dictionaries, making one
         of them the default dictionary.
         Then the reimplementation of the \texttt{bindtextdomain()}
         is called to create a new domain-to-directory binding, and
         it is inserted into a container.
         This container supports looking up the locale directory
         associated with a domain.
% We need to see if we really use nlsbindings!!!!!
      \item \texttt{po::textdomain()}
         This function sets the current domain for the dictionary
         manager to use.
   \end{enumber}

\subsection{Marking a Module for Translation}
\label{subsubsec:potext_usage_marking}

   The basic usage to \textsl{Potext} is essentially identical to
   that of \textsl{GNU Gettext}, except that (currently) only
   \texttt{.po} files are used directly.

   Add the following header file.

   \begin{verbatim}
      #include "po/potext.hpp"      // includes "po/gettext.hpp"
   \end{verbatim}

   Mark each translatable string as usual, using the
   \texttt{\_()} macro:

   \begin{verbatim}
      std::string errmsg = _("Unknown system error");
   \end{verbatim}

   That macro hides a call to \texttt{po::gettext()}.
   Additional "get-text" functions are listed in the table in the
   following section:
   \sectionref{subsec:gettext_module}.

\subsection{Creating the .po Files}
\label{subsubsec:potext_usage_creating_po_files}

   After marking the files that will provide translations, they must be
   processed to extract the marked strings for translation.
   For example:
   
   \begin{verbatim}
      $ xgettext test_helpers.cpp --keyword="_" --output="es.po"
   \end{verbatim}

   The result is an untranslated template, \texttt{es.po}.

   \begin{verbatim}
      # SOME DESCRIPTIVE TITLE.
      # Copyright (C) YEAR THE PACKAGE'S COPYRIGHT HOLDER
      # This file is distributed under the same license as the PACKAGE package.
      # FIRST AUTHOR <EMAIL@ADDRESS>, YEAR.
      #
      #, fuzzy
      msgid ""
      msgstr ""
      "Project-Id-Version: PACKAGE VERSION\n"
      "Report-Msgid-Bugs-To: \n"
      "POT-Creation-Date: 2024-03-20 06:53-0400\n"
      "PO-Revision-Date: YEAR-MO-DA HO:MI+ZONE\n"
      "Last-Translator: FULL NAME <EMAIL@ADDRESS>\n"
      "Language-Team: LANGUAGE <LL@li.org>\n"
      "Language: \n"
      "MIME-Version: 1.0\n"
      "Content-Type: text/plain; charset=CHARSET\n"
      "Content-Transfer-Encoding: 8bit\n"

#: test_helpers.cpp:79
      msgid "output"
      msgstr ""
       . . .
   \end{verbatim}

   The next step is to edit this file appropriately, as in the following
   snippet:

   \begin{verbatim}
      # Mensajes en español para test_helpers.
      # Copyright (C) 2024 Potext Software Foundation Inc.
      # This file is distributed under the same license as the test_helpers package.
      # Chris Ahlstrom <ahlstromcj@gmail.com>, 2024.
      msgid ""
      msgstr ""
      "Project-Id-Version: Potext test_helpers 0.1.0\n"
      "Report-Msgid-Bugs-To: ahlstromcj@gmail.com\n"
      "POT-Creation-Date: 2023-09-18 22:55+0200\n"
      "PO-Revision-Date: 2024-03-20 17:08+0200\n"
      "Last-Translator: Google Translate <translate.google.com>\n"
      "Language-Team: Spanish <es@tp.org.es>\n"
      "Language: es\n"
      "MIME-Version: 1.0\n"
      "Content-Type: text/plain; charset=UTF-8\n"
      "Content-Transfer-Encoding: 8-bit\n"
      "X-Bugs: Report translation errors to the Language-Team address.\n"
      "Plural-Forms: nplurals=2; plural=(n != 1);\n"

      #: test_helpers.cpp:79
      msgid "output"
      msgstr "producción"
       . . .
   \end{verbatim}

   In the project tree, create a \texttt{po} directory and move the
   \texttt{.po} file to it.

   Note that the \textsl{GNU Gettext} manual (\cite{gettextman}) (in chapter
   6) describes many more facets (heh heh) to the creation and manipulation
   of \texttt{.po} files.

\subsection{Installing the .po Files}
\label{subsubsec:potext_usage_installing_po_files}

   The installation process for the project should include installing the
   \texttt{.po} files.
   Where to install them in the system, if desired?
   It does not quite make sense to store them in a place like

   \begin{verbatim}
      /usr/local/share/locale/LL/LC\_MESSAGES}
   \end{verbatim}

   because that contains \texttt{.mo} files (and some \textsl{Qt}
   \texttt{.qm} files!).

   We would suggest something like
   \texttt{/usr/local/share/po/PACKAGE}.
   We should provide support in \textsl{Potext} for that.
   Once \textsl{Potext} supports parsing \texttt{.mo} files,
   the usual processes and location can be used.
   Future stuff.

   The project, once installed, can also, if desired, copy
   the relevant language file to the user's configuration directory,
   \texttt{/home/user/.config/package/} or
   \texttt{C:/Users/user/AppData/Local} at the first run, and use it
   there.

%-------------------------------------------------------------------------------
% vim: ts=3 sw=3 et ft=tex
%-------------------------------------------------------------------------------
