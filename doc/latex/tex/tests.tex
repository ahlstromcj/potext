%-------------------------------------------------------------------------------
% tests
%-------------------------------------------------------------------------------
%
% \file        tests.tex
% \library     Documents
% \author      Chris Ahlstrom
% \date        2024-02-09
% \update      2024-03-15
% \version     $Revision$
% \license     $XPC_GPL_LICENSE$
%
%     Provides a pointed description of some tests, which also helps understand
%     usage of the potext library, a replacement for the tinygettext library..
%
%-------------------------------------------------------------------------------

\section{Potext Tests}
\label{sec:potext_tests}

   This section provides a useful walkthrough of the testing
   of the \textsl{potext} library.
   They illustrate the various way in which the \textsl{Potext} library
   can be used by a develper.

   These tests are not installed; they are in the transitory
   directory \texttt{potext/build/library/tests}.
   The test \texttt{*.po} files are in the 
   directory \texttt{potext/library/tests} and its subdirectories.
   The shell script
   \texttt{library/tests/tests.sh} runs all of the
   \texttt{potext\_test} tests listed in the following file,
   plus a couple more.

   \begin{verbatim}
      library/tests/testlines.list
   \end{verbatim}

   It specifies the tests
   described here, but using only single-word phrases.
   The reason is that we have not yet figured out how to deal
   with phrases enclosed in double-quotes in a shell script.

   Let us survey the main features of all of the test \texttt{po}
   files:

   \begin{itemize}
      \item \texttt{po} directory.
         This directory contains a small sampling of \texttt{.po}
         files from the \textsl{GNU Gettext} project.
         They have been pared down a bit just to save a few bytes,
         and a few extra translations have bee added for testing.
         They are used in the new
         \texttt{hellopotext} test program to
         test the functions in the \texttt{gettext} C++ module.
      \item \texttt{library/tests/de.po}.
         A basic \texttt{po} file with just
         \texttt{msgid} keys and
         \texttt{msgstr} translations in Deutsche (German, Alemania).
      \item \texttt{library/tests/broken.po}.
         This file has an entry \texttt{msgstr[10]}, obviously bad.
      \item \texttt{library/tests/helloworld/de.po}.
         It has some \texttt{msgctxt} sections with
         \texttt{msgid\_plural},
         \texttt{msgstr[0]} for a singular translation and
         \texttt{msgstr[1]} for a plural translation 
         for each context (none, "gui", and "console").
      \item \texttt{library/tests/level/de.po}.
         Contains basic entries plus a number of entries with a blank
         message-ID followed by a long description and a message-string
         with a blank value followed by a long translation.
         NEED TO FIGURE THAT OUT.
         Also includes a couple of \texttt{printf()} format statements.
      \item \texttt{library/tests/po/de.po}.
         Another basic file with a number of translations and a
         weird message-ID called "umlaut".
      \item \texttt{library/tests/po/de\_AT.po}.
         A short file with "umlaut" and a couple of plurals.
      \item \texttt{library/tests/po/fr.po}.
         Contains one German translation. Wtf?
   \end{itemize}

   One thing to watch for in running the tests.
   The test programs write output to \texttt{std::cout} or
   \texttt{std::cerr}, while the \textsl{Potext} library
   internals use the \texttt{po::logstream} class functions.
   What happen is the all of the application output comes first,
   while the log-stream message are not emitted until
   test program exit.
   Not sure why the latter aren't "flushed" immediately.

\subsection{Hello Potext}
\label{subsec:potext_hello_potext}

   This test is a work in progress.
   Without arguments, it runs through basic tests of the following
   functions.
   The main domain is provided to \texttt{po::init\_app\_locale()}
   which should normally be called in the applications's
   \texttt{main()} entry point function.

   \begin{itemize}
      \item \texttt{\_()} and \texttt{gettext()}.
         This function does a message translation lookup using
         the current domain, which is logged in the
         \texttt{init\_app\_locale()} function.
         This smoke test illustrates the most common case we want to
         cover, which is the translation of a phrase according to
         the main (or only) domain and locale directory loaded.
         Also included is a test where the original message is not
         present in the \texttt{.po} file.
      \item \texttt{dgettext}.
         This function looks up a domain's dictionary and uses it
         for the translation.
         This test runs through all of the domains (i.e. \texttt{.po}
         files) in the \texttt{po} project directory.
         Currently tested are the \textsl{es}, \textsl{fr}, \textsl{de},
         and a bogus domain named \textsl{xx}, which should just return
         the input message ID.
      \item \texttt{dcgettext}.
         This test does not do anything.
         Currently \textsl{Potext} does not handle locale categories.
         The reasons? First, the most common use case is looking up
         message translations in the \texttt{LC\_MESSAGES} locale category.
         Second, this translation would require loading additional locale
         directories and their dictionaries.
         With this complication, we will sit on this problem for awhile.
      \item \texttt{ngettext}.
         This test handles plural forms in the current domain.
         It deals only the the main domain, \texttt{es},
         It tests translating the plurals of the following singulars:
         "File" and "Person". 
         The translations are likely not accurate, as they were provided
         by \textsl{Google Translate}.
         But they adequately test the process.
      \item \texttt{dngettext}.
         This test handles plural forms in a specified domain.
         Currently tested are the \textsl{es}, \textsl{fr}, \textsl{de},
         and a bogus domain named \textsl{xx}, which should just return
         the input message ID.
      \item \texttt{dcngettext}.
         This function is not yet implemented, due to difficulties
         with selecting the category directory, as discussed above.
      \item \texttt{pgettext}.
         This test applies only to the domain specified in the
         \texttt{init\_app\_locale()} function.
   \end{itemize}

   Some functions not yet tested because of the implmentation difficulties
   noted above.

   \begin{itemize}
      \item \texttt{dcgettext()}.
      \item \texttt{dcngettext()}.
   \end{itemize}

   Additional arguments can change the default domain.
   We will document these real soon now.

\subsection{Potext Test Program}
\label{subsec:potext_test_program}

   This small application is the best test of \textsl{potext} from the
   perspective of a developer wanting to use it in an application.
   Running it without any arguments shows a list of 8 tests.
   These are reflected in the following sections.

\subsubsection{Potext Test 'Translate'}
\label{subsubsec:potext_test_translate}

   These tests are run using a command like the following:

   \begin{verbatim}
      $ ./build/library/tests/potext_test translate <...options...>
   \end{verbatim}

   By running \texttt{potext\_test} without any options, one sees
   four "translate" commands.
   The four variations on the "translate"
   test are described in the following sub-sections.

\paragraph{Translate Basic: Po File and Sample Message}
\label{paragraph:potext_test_translate_po_message}

   This test is a simple translation of a word.
   The basic "translate" test is run by the following form, which has an
   argument count of 4.

   \begin{verbatim}
      $ ./build/library/tests/potext_test translate <file> <msg>
   \end{verbatim}

   The file is a test \texttt{.po} file in the
   \texttt{tests} directory.
   The message is a phrase present in that file, such as
   \textsl{"F1 - show/hide this help screen"}, translated in
   \texttt{de.po} as
   \textsl{"F1 - Hilfe anzeigen/verstecken"}.

   Here is the run:

   \begin{verbatim}
      $ ./build/library/tests/potext_test translate \
         library/tests/de.po "F1 - show/hide this help screen"
   \end{verbatim}

   The output is

   \begin{verbatim}
      Translation: "F1 - Hilfe anzeigen/verstecken"
   \end{verbatim}

   If only a part of the message is provided, of course there is no
   match, and the message is

   \begin{verbatim}
      Translation: "F1 - Hilfe"
      [potext] Couldn't translate: "F1 - Hilfe"
   \end{verbatim}

   This second test shows that any deviation from a supported message causes an
   warning, and returns the original message.
   These deviations include missing letters, missing punctuation,
   additional spaces.
   \textsl{We wonder if we can work around such issues in this library.
   We shall see.}

\paragraph{Translate: Po File, Context, and Sample Message}
\label{paragraph:potext_test_translate_po_context_message}

   This test is run by the following form, which has an argument count
   of 5.

   \begin{verbatim}
      $ ./build/library/tests/potext_test translate <file> <context> <msg>
   \end{verbatim}

   This test requires a \texttt{po} file with message-context
   entries such as these three different entries found in
   \texttt{library/test/helloworld/de.po} for the phrase
   "Hello World":

   \begin{verbatim}
      msgctxt ""
      msgid "Hello World"
      msgctxt "console"
      msgid "Hello World"
      msgctxt "gui"
      msgid "Hello World"
   \end{verbatim}

   Please note that the \textsl{GNU} \texttt{gettext()} documentation says
   that an empty message context (\texttt{msgctxt ""}) is \textsl{not}
   the same as a missing message context.
   In the "helloworld" test program, these contexts are provided by the
   following lines:

   \begin{verbatim}
      pgettext("", "Hello World")
      pgettext("console", "Hello World")
      pgettext("gui", "Hello World")
   \end{verbatim}

   The macros \texttt{ngettext} and \texttt{npgettext} are also used to provide
   access to the various plural forms in that \texttt{po} file.
   In any case, we need to use 
   \texttt{library/test/helloworld/de.po} for this test.
   An actual test run:

   \begin{verbatim}
      $ ./build/library/tests/potext_test translate \
            library/tests/helloworld/de.po "console" "Hello World"
   \end{verbatim}

   The output is

   \begin{verbatim}
      Context 'console' translation: "Hallo Welt (singular) in der Console"
   \end{verbatim}

   If the \texttt{<context>} parameter is not found in the \texttt{po}
   file, a message is emitted to indicate the error.

\paragraph{Translate: Po File with Singular and Plurals}
\label{paragraph:potext_test_translate_po_singular_plural}

   This test is run by the following form, which has an argument count
   of 6.

   \begin{verbatim}
      $ ./build/library/tests/potext_test translate <file> <singular>
            <plural> <N>
   \end{verbatim}

   The singular and plural parameters are message IDs, such as "Hello World".
   This command is a bit tricky; the N value is not a C index, but an
   index that starts at 1.
   The N parameter ranges from 1 to the last array value in
   the \texttt{po} file.
   The number of singular/plural translations depends on the language and
   is specified in the specific \texttt{.po} file using
   a header declaration such as
   \texttt{"Plural-Forms: nplurals=2; plural=(n != 1);"}.
   Look at \texttt{pluralforms.cpp} to see all the plural-forms settings and
   "callbacks" that are support.
   Some of these forms support Slavic and Arabic languages, and we are not able
   to test them.

   An actual test run:

   \begin{verbatim}
      $ ./build/library/tests/potext_test translate \
            library/tests/helloworld/de.po "Hello World" "Hello Worlds" 1
   \end{verbatim}

   The output is

   \begin{verbatim}
      TODO
   \end{verbatim}

\paragraph{Translate: Po File with Context, Singular, and Plurals}
\label{paragraph:potext_test_translate_po_context_singular_plural}

   This test is run by the following form, which has an argument count
   of 7.

   \begin{verbatim}
      $ ./build/library/tests/potext_test translate <file> <context> \
         <singular> <plural> <N>
   \end{verbatim}

   An actual test run:

   \begin{verbatim}
      $ ./build/library/tests/potext_test translate \
            library/tests/helloworld/de.po "gui" "Hello World" "Hello Worlds" 0
   \end{verbatim}

   The output is

   \begin{verbatim}
      TODO
   \end{verbatim}

\subsubsection{Potext Test 'Directory'}
\label{subsubsec:potext_test_directory}

   These tests are run using a command like the following:

   \begin{verbatim}
      $ ./build/library/tests/potext_test directory <dir> <msg> [<lang>]
   \end{verbatim}

\subsubsection{Potext Test 'Language'}
\label{subsubsec:potext_test_language}

   These tests are run using a command like the following:

   \begin{verbatim}
      $ ./build/library/tests/potext_test language <lang>
   \end{verbatim}

\subsubsection{Potext Test 'Language Directory'}
\label{subsubsec:potext_test_language_directory}

   These tests are run using a command like the following:

   \begin{verbatim}
      $ ./build/library/tests/potext_test language-dir <dir>
   \end{verbatim}

\subsubsection{Potext Test 'List Message Strings'}
\label{subsubsec:potext_test_list_message_strings}

   These tests are run using a command like the following:

   \begin{verbatim}
      $ ./build/library/tests/potext_test list-msgstrs <file>
   \end{verbatim}

%-------------------------------------------------------------------------------
% vim: ts=3 sw=3 et ft=tex
%-------------------------------------------------------------------------------
